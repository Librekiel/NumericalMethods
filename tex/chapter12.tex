\section{Теоретический материал к семинару №12}
Для решения гиперболического уравнения
\begin{equation} \label{c12eq1}
	\begin{cases}
		\displaystyle \frac{\partial^2 u}{\partial t^2} = c^2 \frac{\partial^2 u}{\partial x^2} + f(x,t) \\
		u(0,t) = \mu_1(t) \\
		u(a,t) = \mu_2(t) \\
		u(x,0) = \mu_3(x) \\
		u_t(x,0) = \mu_4(x)
	\end{cases}
\end{equation}
в области $(x,t) \in [0; a] \times [0; T]$ можно использовать «схему с весами» 
\begin{equation} \label{c12eq2}
	\frac{1}{\tau^2} \left( \hat{u} - 2u + \check{u} \right) = \Lambda \left( \sigma \hat{u} + \left( 1 - 2 \sigma \right) u + \sigma \check{u} \right) + f,
\end{equation}
где $\Lambda$ - оператор пространственного дифференцирования (с учетом умножения на $c^2$).

Выражение \eqref{c12eq2} можно преобразовать так:
\begin{equation} \label{c12eq3}
	\left( E - \sigma \Lambda^* \right) \left( \hat{u} - 2u - \check{u} \right) = \Lambda^* u,
\end{equation}
где $\Lambda^* = \tau^2 \Lambda$. Теперь нетрудно выразить решение на текущем слое через два предыдущих слоя
\begin{equation} \label{c12eq4}
	\hat{u} = \left( E - \sigma \Lambda^* \right)^{-1} \Lambda^* u + 2u - \check{u}
\end{equation}
Для вычисления решения на первом слое надо использовать следующую формулу:
\begin{equation} \label{c12eq5}
	\begin{split}
		u(x,\tau) ??? u(x,0) + \tau u_t + \frac{\tau^2}{2} u_{tt} = \\
		= u(x,0) + \tau \mu_4 + \frac{\tau^2}{2} \left(c^2 u_{xx} + f \right) ??? u(x,0) + \tau \mu_4 + \frac{1}{2} \left( \Lambda^* \mu_3 + \tau^2 f \right).
	\end{split}
\end{equation}
В случае независящих от времени граничных условий Дирихле удобнее всего обеспечить их выполнение, занулив первую и последнюю строку в матрице оператора $\Lambda^*$. Именно так это делалось ранее при решении уравнения теплопроводности с аналогичными граничными условиями.

Для контроля правильности счета следует провести несколько расчетов со сгущением сетки (сгущать сетку необходимо как по пространству, так и по времени) и вычислить эффективный порядок метода. Поскольку теоретическая оценка погрешности метода $O\left( \tau^2 + h^2 \right)$, то при одновременном сгущении сетки по пространству и времени в одно и то же число раз должен получиться второй порядок.

Сгущать сетку удобнее всего каждый раз вдвое, а погрешность рассчитывать  по общим узлам двух соседних вложенных сеток. Наиболее удобная норма в данном случае – чебышевская, или норма $C$. Формула расчета эффективного порядка по трем сеткам выглядит так:
\begin{equation} \label{c12eq6}
	p = - \log_2 \frac{\| U_{4N} - U_{2N} \|_C}{\| U_{2N} - U_N \|_C}.
\end{equation}
Разности вычисляются по общим узлам двух сеток.

\subsection{Задание к семинару №12}
\begin{enumerate}
\item Решить задачу \eqref{c12eq1} при $f = 0$, $c = 3$, $\mu_1 = \mu_2 = 0$, $\mu_3(x) = \sin (x)$, $\mu_4 = 0$ в области $[0; 6\pi] \times [0; 10]$. Взять $\tau = 0.01$ и $h = 6\pi/100$. Отобразить решение на каждом временном слое.
\item В условии 1 задания взять 
\begin{equation} \nonumber
	\mu_3(x) = \sin \left( x \left( 1 + 0.1 e^{-(x-10)^2}\right) \right),
\end{equation}
подобрать согласованные граничные условия. Все остальное оставить как в 1 задании. Повторить расчет.
\item В условии 2 задания провести расчет на сгущающихся сетках и доказать второй порядок метода. Задачу решать в области $[0; 6\pi] \times [0; 1]$, для первой сетки взять $\tau = 1/16$ и $h = 6\pi/16$. Провести расчеты на 7 сетках. Решение на каждом слое не отображать. Построить график эффективного порядка от номера самой грубой сетки из трех сеток, участвующих в расчете эффективного порядка.
\end{enumerate}

