\section{Теоретический материал к семинару №4}
Дифференциально-алгебраическая система содержит как дифференциальные, так и алгебраические уравнения. В общем виде она записывается как
\begin{equation} \label{c4eq1}
	\begin{cases}
		\displaystyle \mathbf{G} \frac{d\mathbf{u}}{dt} = \mathbf{F} \left( \mathbf{u}, t \right) \\
		\mathbf{u}(t_0) = \mathbf{u}_0
	\end{cases}
\end{equation}
где $\mathbf{G}$ - матрица коэффициентов при производных. 

Для ее решения можно использовать семейство схем Розенброка, заменив в них матрицу $\mathbf{E}$ матрицей $\mathbf{G}$  
\begin{equation} \label{c4eq2}
	\begin{cases}
		\displaystyle \left( \mathbf{G} - \alpha \tau \mathbf{F_u} \right) \boldsymbol{\omega} = \mathbf{F} \left( \mathbf{u}, t + \frac{\tau}{2} \right) \\
		\mathbf{\hat{u}} = \mathbf{u} + \tau \operatorname{Re} \boldsymbol{\omega}
	\end{cases}.
\end{equation}
Здесь $\alpha$ - параметр схемы, $\tau$ - шаг схемы по времени, $\mathbf{F_u}$ - производная правой части по переменной $\mathbf{u}$, $\mathbf{u}$ и $\mathbf{\hat{u}}$ - численное решение в текущий и следующий моменты времени соответственно.

Для получения второго порядка в одностадийной схеме Розенброка для задачи Коши необходимо брать $\displaystyle \alpha = \frac{1}{2}$ или $\displaystyle \alpha = \frac{1+i}{2}$. Для неавтономных дифференциально-алгебраических систем даже при этих   получить второй порядок точности не удается из-за противоречивых требований к выбору смещения по времени при вычислении правой части. Поэтому для получения второго порядка необходимо провести автономизацию, то есть убрать явную зависимость от времени в правой части.

\subsection{Задачи к семинару №4}
\begin{enumerate}
\item Требуется рассчитать работу транзисторного усилителя. Для этого необходимо решить дифференциально-алгебраическую систему.

Параметры электрической схемы:
\begin{matlablisting}
	\begin{lstlisting}
global r0 r1 r2 r3 r4 r5 ub;
r0 = 1000; r1 = 9000; r2 = r1; r3 = r1; r4 = r1; r5 = r1;
c1 = 1e-6; c2 = 2e-6; c3 = 3e-6; ub = 6;
	\end{lstlisting}
\end{matlablisting}
Начальные условия:
\begin{matlablisting}
	\begin{lstlisting}
u0 = [0; ub*r1/(r1+r2); ub*r1/(r1+r2); ub; 0];
	\end{lstlisting}
\end{matlablisting}
Матрица коэффициентов при произоводных:
\begin{matlablisting}
	\begin{lstlisting}
G = [-c1  c1  0   0   0;
      c1 -c1  0   0   0;
      0   0  -c2  0   0;
      0   0   0  -c3  c3;
      0   0   0   c3 -c3];
	\end{lstlisting}
\end{matlablisting}
Правая часть:
\begin{matlablisting}
	\begin{lstlisting}
function y = ue(t)
    y = 0.1*sin(200*pi*t);
end

function y = ff(u)
    y = 1e-6*(exp(u/0.026)-1);
end

function y = F(t, u)
    global r0 r1 r2 r3 r4 r5 ub;
    y = [u(1)/r0-ue(t)/r0; 
         0.01*ff(u(2)-u(3))-ub/r2+u(2)*(1/r1+1/r2);
         u(3)/r3-ff(u(2)-u(3)); 
         0.99*ff(u(2)-u(3))-ub/r4+u(4)/r4; 
         u(5)/r5                                  ];
end
	\end{lstlisting}
\end{matlablisting}
Расчет провести с шагом $h = 1/5000$ на временном отрезке от 0 до 0.3 при $\displaystyle \alpha = \frac{1}{2}$ и $\displaystyle \alpha = \frac{1+i}{2}$. Вывести результат расчета на график (на одном графике будет 2 семейства кривых для двух разных $\alpha$, в каждом семействе по 5 кривых, соответствующих 5 компонентам $\mathbf{u}$). Объяснить разницу между численными решениями при разных $\alpha$.
\item Определить  эффективный порядок метода с помощью сгущения сеток. Для экономии времени расчет вести до $t = 0.01$. Провести автономизацию и вновь определить эффективный порядок.
\end{enumerate}