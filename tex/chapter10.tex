\section{Задание к семинару №10}
Решить двумерное однородное уравнение теплопроводности с граничными условиями Дирихле 
\begin{equation} \label{c10eq1}
	\begin{cases}
		\displaystyle \frac{\partial u}{\partial t} = k \left( \frac{\partial^2 u}{\partial x^2} + \frac{\partial^2 u}{\partial y^2} \right) \\
		\displaystyle u(x,y,0) = \mu(x,y) \\
		u(0,y,t) = \mu(0,y), \qquad u(a,y,t) = \mu(a,y)\\  
		u(x,0,t) = \mu(x,0), \qquad u(x,b,t) = \mu(x,b)
	\end{cases}
\end{equation}
в области $(x,y,t) \in [0; a] \times [0; b] \times [0; T]$.

Функция $\mu(x,y)$ задается следующим образом ($\texttt{\lstinline|x|}$ и $\texttt{\lstinline|y|}$ - вектора, содержащие координаты узлов прямоугольной сетки):
\begin{matlablisting}
	\begin{lstlisting}
function z = mu(x, y)
    z = zeros(length(y), length(x));
    for i = 1:length(x)
        for j = 1:length(y)
            z(j,i) = -0.01*sin(x(i)) + 0.05*sin(y(j));
        end
    end
end
	\end{lstlisting}
\end{matlablisting}
Параметры области: $a = 6\pi$, $b = 4\pi$, $T = 10$. Выбрать равномерную сетку с $h_x = h_y = \pi/30$ и шагом по времени $\tau = 0.1$. Коэффициент теплопроводности $k = 0.2$. Отображать двумерное решение на каждом временном слое с помощью функции $\texttt{\lstinline|mesh|}$.

\subsection{Указания}
Для решения использовать эволюционно-факторизованную схему
\begin{equation} \label{c10eq2}
	\begin{cases}
		\displaystyle \left( E - \frac{\tau}{2} \Lambda_x \right) \nu = \Lambda_x u + \Lambda_y u \\
		\displaystyle \left( E - \frac{\tau}{2} \Lambda_y \right) \Delta u = \nu \\
		\hat{u} = u + \tau \Delta u
	\end{cases}.
\end{equation}
Здесь $\Lambda_x$ и $\Lambda_y$ - операторы пространственного дифференцирования. Проблем с промежуточным граничным условием здесь не возникает, так как $u$ на границе не меняется со временем, а значит на границе $\hat{u} - u = 0$ и $\nu = 0$.

Операторы $P_x = E - \frac{\tau}{2} \Lambda_x$ и $P_y = E - \frac{\tau}{2} \Lambda_y$ могут быть записаны в матричной форме с помощью замены $\Lambda_x$ и $\Lambda_y$ на соответствующие матрицы пространственного дифференцирования $L_x$ и $L_y$. При этом надо помнить, что
\begin{equation} \label{c10eq3}
	\begin{split}
		\Lambda_x u = u L_x \\
		\Lambda_y u = L_y u
	\end{split},
\end{equation}
т.е. матрицы пространственного дифференцирования $L_x$ и $L_y$ умножаются на матрицу значений в узлах сетки $u$ с  разных сторон. Аналогично при обращении операторов $P_x$ и $P_y$ надо использовать правое и левое матричное деление соответственно.

Не следует применять операторы пространственного дифференцирования $\Lambda_x$ и $\Lambda_y$ к крайним строкам и столбцам матрицы $u$, так как в таком случае в $\Delta u$ в крайних строках и столбцах будут содержаться ненулевые значения, что приведет к нарушению граничных условий в расчете.


