\section{Теоретический материал к семинару №2}
Схема Рунге-Кутты для решения задачи Коши
\begin{equation} \label{c2eq1}
	\begin{cases}
		\displaystyle \frac{d\mathbf{u}}{dt} = \mathbf{f} \left( \mathbf{u}, t \right) \\
		\mathbf{u}(t_0) = \mathbf{u}_0
	\end{cases}
\end{equation}
имеет следующий вид:
\begin{equation} \label{c2eq2}
	\begin{split}
		\mathbf{u}_{n+1} = \mathbf{u}_n + \tau_n \sum_{k=1}^{s} b_k \boldsymbol{\omega}_k, 
	\qquad
		\tau_n = t_{n+1} - t_n; \\
		\boldsymbol{\omega}_k = \mathbf{f} \left( \mathbf{u}_n + \tau_n \sum_{l=1}^{L} \alpha_{kl} \boldsymbol{\omega}_l, t_n + \tau_n a_k \right),
	\qquad
		1 \le k \le s.
	\end{split}
\end{equation}
Здесь $\tau_n$ - шаги по времени, $s$ - число стадий, коэффициенты $\alpha_{kl}$ образуют матрицу Бутчера $\mathbf{A}$, a $a_k$ и $b_k$ - элементы векторов $\mathbf{a}$ и $\mathbf{b}$, вместе с матрицой Бутчера полностью задающих схему Рунге-Кутта.

Для реализации на компьютере в системе MATLAB удобнее записать \eqref{c2eq2} в векторной форме
\begin{equation} \label{c2eq3}
	\begin{split}
		\mathbf{u}_{n+1} = \mathbf{u}_n + \tau_n \boldsymbol{\omega} \mathbf{b}^T,
	\qquad
		\tau_n = t_{n+1} - t_n; \\
		\boldsymbol{\omega}_k = \mathbf{f} \left( \mathbf{u}_n + \tau_n \boldsymbol{\omega} \boldsymbol{\alpha}^T_k, t_n + \tau_n a_k \right),
	\qquad
		1 \le k \le s,
	\end{split} 
\end{equation}
где $\boldsymbol{\omega}_k$ - $k$-тая строка матрицы промежуточных результатов $\boldsymbol{\omega}$, $\mathbf{b}$ - вектор-строка коэффициентов $b$ и $\boldsymbol{\alpha}_k$ - $k$-тая строка матрицы Бутчера. Верхний индекс $T$ означает транспанирование.

\subsection{Задачи к семинару №2}
\begin{enumerate}
\item Записать расчетные формулы для схемы Кутта, если
\begin{equation} \nonumber
	\mathbf{A} = 
		\begin{pmatrix}
		0 & 0 & 0 & 0 \\
		1/2 & 0 & 0 & 0 \\
		0 & 1/2 & 0 & 0 \\
		0 & 0 & 1 & 0 \\
		\end{pmatrix},
	\qquad
	\mathbf{a} = 
		\begin{pmatrix}
		0 \\
		1/2 \\
		1/2 \\
		1 \\
		\end{pmatrix},
	\qquad
	\mathbf{b} = 
		\begin{pmatrix}
		1/6 \\
		1/3 \\
		1/3 \\
		1/6 \\
		\end{pmatrix}^T.
\end{equation}
\item Перейти к длине дуги в задаче
\begin{equation} \nonumber
	\begin{cases}
		\displaystyle \frac{du}{dt} = u^2 + t^2 \\
		u(t_0) = u_0
	\end{cases}.
\end{equation}
\item Реализовать схему Кутта на компьютере.
\begin{enumerate}
\item Правая часть: 
\begin{matlablisting}
	\begin{lstlisting}
function y = f(t, u)
    y = u + t^2 + 1;
end
	\end{lstlisting}
\end{matlablisting}
Начальное условие:
\begin{matlablisting}
	\begin{lstlisting}
u0 = 0.5;
	\end{lstlisting}
\end{matlablisting}
\item Правая часть:
\begin{matlablisting}
	\begin{lstlisting}
function y = f(t, u)
    om = [sin(t) cos(t) sin(t+pi/4)];
    Omega = [   0   -om(3)  om(2); 
              om(3)    0   -om(1);
             -om(2)  om(1)    0   ];
    y = Omega * u;
end
	\end{lstlisting}
\end{matlablisting}
Начальное условие:
\begin{matlablisting}
	\begin{lstlisting}
u0 = [1; -0.5; 0.6];
	\end{lstlisting}
\end{matlablisting}
\end{enumerate}
Временной отрезок для обеих функций - от 0 до 1. Провести 7 расчетов на сгущающихся вдвое сетках, начиная с минимально возможной сетки из 1 интервала. Для первой функции построить график эффективного порядка метода от числа интервалов сетки (по последнему узлу, т.е. в последнем узле сетки при $t=1$), для второй - построить график решения (3 кривые на одном графике). 
\item Реализовать явную схему Рунге-Кутты в общем виде. Для отладки использовать 7-стадийную схему Хаммуда 6 порядка:
\begin{matlablisting}
	\begin{lstlisting}
butcher = [           0                      0            ...
                      0                      0            ...
                      0                      0             0;
                     4/7                     0            ...
                      0                      0            ... 
                      0                      0             0;
                   115/112                 -5/16          ...  
                      0                      0            ...
                      0                      0             0;
                   589/630                  5/18          ...
                   -16/45                    0            ...
                      0                      0             0;
           229 /1200-29/6000*5^0.5 119/240-187/1200*5^0.5 ...
             -14/75+34/375*5^0.5       -3/100*5^0.5       ...        
                      0                      0             0;
           71/2400-587/12000*5^0.5 187/480-391/2400*5^0.5 ...
             -38/75+26/375*5^0.5     27/80-3/400*5^0.5    ...
                 (1+5^0.5)/4                 0             0;
            -49/480+43/160*5^0.5    -425/96+51/32*5^0.5   ...   
               52/15-4/5*5^0.5       -27/16+3/16*5^0.5    ...
                5/4-3/4*5^0.5          5/2-0.5*5^0.5       0 ];

a = [0 4/7 5/7 6/7 (5-5^0.5)/10 (5+5^0.5)/10 1];
b = [1/12 0 0 0 5/12 5/12 1/12];
	\end{lstlisting}
\end{matlablisting}
Провести 7 расчетов на сгущающихся вдвое сетках, начиная с минимально возможной сетки из 1 интервала. Протестировать на тех же тестовых функциях, построить такие же графики.
\end{enumerate}