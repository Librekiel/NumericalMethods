\section{Теоретический материал к семинару №1}

\subsection{Оценка погрешности по Ричардсону}
Пусть некоторая величина $U$ была вычислена на сетке с шагом $h$, тогда
\begin{equation} \label{c1eq1}
	U = U_h + R_h,
\end{equation}
где $U_h$ - вычисленное значение, а $R_h$ - его погрешность. Теперь изменим шаг сетки в $r$ раз. Получим аналогично
\begin{equation} \label{c1eq2}
	U = U_{rh} + R_{rh}.
\end{equation}
Если метод сходится к точному решению с порядком $p$, то
\begin{equation} \label{c1eq3}
	R_h = ch^p + o(h^p),
\end{equation}
где $c$ - некоторая константа, независящая от $h$, причем $c \neq 0$.
Записав аналогичное выражение для погрешности расчета с шагом $rh$ и подставив результат в \eqref{c1eq2}, а также подставив \eqref{c1eq3} в \eqref{c1eq1} получим
\begin{equation} \label{c1eq4}
	U = U_h + ch^p + o(h^p),
	\qquad 
	U = U_{rh} + cr^ph^p + o(h^p).
\end{equation}
Пренебрегая членами более высокого порядка, чем $h^p$, вычтем из первого уравнения второе, тогда
\begin{equation} \label{c1eq5}
	0 = U_h - U_{rh} + ch^p(1-r^p).
\end{equation}
Выразим из этого равенства $ch^p$
\begin{equation} \label{c1eq6}
	ch^p = \frac{U_{rh} - U_h}{1 - r^p}.
\end{equation}
Если отбросить члены более высокого, чем $p$ порядка, то $R_{rh} = cr^ph^p$. Отсюда, учитывая \eqref{c1eq6}, получим
\begin{equation} \label{c1eq7}
	R_{rh} = \frac{U_{rh} - U_h}{r^{-p} - 1}.
\end{equation}
Эта формула дает асимптотически точную оценку погрешности на более подробной сетке. Добавив ее к $U_{rh}$, мы можем получить ответ с более высоким порядком точности
\begin{equation} \label{c1eq8}
	U_{rh}^* = U_{rh} + R_{rh} = U_{rh} + \frac{U_{rh} - U_h}{r^{-p} - 1} = \frac{r^{-p}U_{rh} - U_h}{r^{-p} - 1}.
\end{equation}

\subsection{Оценка погрешности по Эйткену}
Недостатком рассмотренного метода вычисления погрешности является необходимость знания порядка метода $p$. Если сделать два последовательных сгущения в одинаковое число раз, то можно оценить порядок. В самом деле, рассмотрим отношение
\begin{equation} \label{c1eq9}
	\frac{U_{r^2h} - U_{rh}}{U_{rh} - U_h} = \frac{\left( 1 - r^p \right) cr^ph^p}{\left( 1 - r^p \right) ch^p} = r^p.
\end{equation}
Здесь было дважды использовано выражение \eqref{c1eq6}. Подставив $r^p$ в выражение
\begin{equation} \label{c1eq10}
	R_{r^2h} = \frac{U_{r^2h} - U_{rh}}{r^{-p} - 1}.
\end{equation}
получим требуемую оценку погрешности
\begin{equation} \label{c1eq11}
	R_{r^2h} = \frac{U_{r^2h} - U_{rh}} {\displaystyle \frac{U_{rh} - U_h}{U_{r^2h} - U_{rh}} - 1}.
\end{equation}
Аналогично \eqref{c1eq8} можно вычислить уточненное решение
\begin{equation} \label{c1eq12} 
	U_{r^2h}^* = U_{r^2h} + R_{r^2h} = U_{r^2h} + \frac{U_{r^2h} - U_{rh}} {\displaystyle \frac{U_{rh} - U_h}{U_{r^2h} - U_{rh}} - 1}.
\end{equation}

\subsection{Рекуррентное сгущение}
Формулы \eqref{c1eq8} и \eqref{c1eq12} позволяют получать уточненные решения. Если имеются $N$ последовательных расчетов на сгущающихся сетках, то группируя 1 и 2, 2 и 3, 3 и 4 и т.д. по формуле \eqref{c1eq8} получим $N-1$ уточненных значений. Для сгущения по Эйткену вместо пар расчетов следует использовать тройки: 1,2 и 3, затем 2,3 и 4 и т.п. При этом получится уже $N-2$ уточненных значений. 

К полученному набору уточненных значений можно вновь применить описанную выше процедуру, но для уточнения по Ричардсону потребуется знание порядка точности уточненных значений, полученных на предыдущем шаге. Для уточнения по Эйткену это не нужно.

Ясно, что этот процесс можно продолжать, пока не останутся два уточненных значения в методе Ричардсона и три – в методе Эйткена.

\subsection{Эффективный порядок метода}
Из формулы \eqref{c1eq9} можно выразить эффективный порядок метода $p$
\begin{equation} \label{c1eq13}
	p = \log_r \frac{U_{r^2h} - U_{rh}}{U_{rh} - U_h}.
\end{equation}
Для вычисления порядка требуется 3 сетки. Если имеются $N$ расчетов, то по 1,2 и 3 расчету можно получить первую оценку порядка, по 2,3 и 4 – вторую и т.д. – всего $N-2$ оценок. Потом можно построить график эффективного порядка, отложив по оси абсцисс номер тройки, а по оси ординат - соответствующее значение $p$. По такому графику можно наблюдать, к какому значению стремится .
Существует еще один способ определения эффективного порядка – построение графика $\lg\left| U_{rh} - U_h \right|$ от $-\lg h$ (или от $\lg N$, где $N$ - число интервалов сетки). Действительно, согласно \eqref{c1eq6}
\begin{equation} \label{c1eq14}
	\lg \left| U_{rh} - U_h \right| = \lg \big( \left| c \right| \left( 1 - r^p \right) \big) + \left( -p \right) \left( - \lg h \right).
\end{equation}
Отсюда видно, что график должен быть близок к прямой, тангенс угла наклона которой даст нам. Угол наклона можно определить, например, методом наименьших квадратов (в системе MATLAB Tools/Basic Fitting/linear в окне графика). 

\textbf{Замечание.} Выше при выводе и конечной записи всех формул использовался шаг сетки $h$ и множитель $r$, означающий отношение шага сгущенной сетки к шагу исходной. Часто более удобно использовать число интервалов сетки $N$, а в качестве множителя $r$ - отношения числа интервалов новой сетки к числу интервалов старой. Очевидно, что «новый» $r$ будет обратной величиной к «старому». Поэтому для перехода от шагов сетки к  числам интервалов необходимо провести следующие замены:
\begin{equation} \label{c1eq15}
	r \rightarrow r^-1, \ U_{rh} \rightarrow U_{rN}, \ U_{r^2h} \rightarrow U_{r^2N}, \ R_{rh} \rightarrow R_{rN}, \ R_{r^2h} \rightarrow R_{r^2N}.
\end{equation}
Например, в этих обозначениях основная формула \eqref{c1eq7} будет выглядеть как
\begin{equation} \label{c1eq16}
	R_{rN} = \frac{U_{rN} - U_N}{r^p - 1}.
\end{equation}

\subsection{Задачи к семинару №1}
\begin{enumerate}
\item Вычислить $\displaystyle f = 21b^2 - 2a^2 + 44b^4 - 10a^2b^2 + \frac{a}{2b}$ при $a = 77617$ и $b = 33096$.
\item Вычислить собственные значения матрицы $A$ аналитически и средствами MATLAB. Отобразить результат на комплексной плоскости.
\begin{equation} \nonumber
	A = 
	\begin{pmatrix}
		1 & 0 & \cdots & \cdots & 0 & 10^{-2m} \\
		1 & 1 & \ddots &  &  & 0 \\
		0 & 1 & \ddots &  \ddots &  & \vdots \\
		\vdots &  & \ddots & \ddots & \ddots & \vdots \\
		\vdots &  &  & \ddots & \ddots  & 0 \\
		0 & \cdots & \cdots &  \cdots & 1 & 1 \\
	\end{pmatrix}_{2m \times 2m}
\end{equation}
Расчет выполнить при $m = 20,23,30$.
\item Получить формулу Симпсона из формулы трапеций с помощью однократного сгущения сетки в 2 раза и формулы Ричардсона \eqref{c1eq7}.
\item Вычислить интеграл 
\begin{equation} \nonumber
	\int_{0}^{1} \frac{4}{1 + x^2} dx
\end{equation}
численно по формуле трапеций при числе интервалов сетки $N = 16,32,64,128,256$. Затем с помощью формулы \eqref{c1eq13} или формулы \eqref{c1eq14} построить график эффективного порядка. Провести уточнение решения по формуле Ричардсона, получив из 5 расчетных значений 4 уточненных. По уточненным значениям вновь построить график эффективного порядка.
\item Вычислить интеграл 
\begin{equation} \nonumber
	\int_{0}^{1} \frac{1}{2 \sqrt{x}} dx
\end{equation}
по формуле средних прямоугольников при тех же $N$. Определить эффективный порядок, провести уточнение по Эйткену (получив 3 уточненных значения), вновь определить эффективный порядок. После этого провести еще одно уточнение решения по Эйткену и сравнить полученный результат с точным значением интеграла.
\end {enumerate}