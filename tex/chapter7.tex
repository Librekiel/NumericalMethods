\chapter{}

\section{Теоретический материал}
Одномерное уравнение переноса имеет следующий вид:
\begin{equation} \label{c7eq1}
	\begin{cases}
		\displaystyle \frac{\partial u}{\partial t} + c \frac{\partial u}{\partial x} = f(x,t) \\
		u(0,t) = \mu_1(t) \\
		u(x,0) = \mu_2(x)
	\end{cases}.
\end{equation}
Здесь $c > 0$ - скорость переноса, $\mu_1(t)$ - граничное, а $\mu_2(x)$ - начальное условия. Аппроксимируя производные конечными разностями, получим ряд разностных схем для этого уравнения:
\begin{align}
	\frac{\hat{u}_n - u_n}{\tau} + c \frac{u_n - u_{n-1}}{h} = \Phi_n, \label{c7eq2} \\
	\frac{\hat{u}_{n-1} - u_{n-1}}{\tau} + c \frac{\hat{u}_n - \hat{u}_{n-1}}{h} = \Phi_n, \label{c7eq3} \\
	\frac{\hat{u}_n - u_n}{\tau} +  c \frac{\hat{u}_n - \hat{u}_{n-1}}{h} = \Phi_n, \label{c7eq4} \\
	\frac{\hat{u}_n + \hat{u}_{n-1} - u_n - u_{n-1}}{\tau} + c \frac{\hat{u}_n + u_n - \hat{u}_{n-1} - u_{n-1}}{h} = 2 \Phi_n, \label{c7eq5}
\end{align}
где $\tau$ - шаг по времени сетки, $h$ - шаг по пространству, а $\Phi_n = f(x_{n-1/2}, t_{n+1/2})$ - значение функции из правой части уравнения в середине ячейки.

Первые две схемы условно устойчивы, две последние – безусловно устойчивы. Граница устойчивости задается с помощью неравенства, в которое входит число Куранта $\kappa = c\tau/h$. Первая схема устойчива при $\kappa \leq 1$, а вторая – при $\kappa \geq 1$. Таким образом, их условия устойчивости противоположны. Это дает возможность построить из них так называемую составную схему, называемую также схемой Карсона. Идея в том, чтобы при $\kappa \leq 1$ использовать первую схему, а при $\kappa \geq 1$ - вторую. В итоге составная схема получается безусловно устойчивой. Если число Куранта $\kappa$ переходит через единицу, например из-за изменения скорости переноса или неравномерности сетки, то часть шагов может быть сделана по первой схеме, а часть - по второй. Оказывается, что составная схема оказывается несколько точнее безусловной устойчивой чисто неявной схемы \eqref{c7eq4}.

Схема \eqref{c7eq2} - явная, все остальные - формально неявные, хотя на самом деле считать по ним не труднее, чем по явным, поскольку все неизвестные величины на следующем временном слое получаются либо из граничного условия, либо из начального, либо из результата предыдущего расчета. Важно лишь соблюдать правильный порядок вычислений - от левой границы области расчета к правой и от более раннего временного слоя к более позднему.

\section{Задание}
Задачу будем решать на отрезке  $[0; 100]$ по пространству и   [0; 1] по времени. Шаг равномерной сетки по пространству $h = 0.1$, шаг по времени $\tau = 0.01$. Скорость переноса $c = 50$. Правая часть - нулевая. Начальное условие
\begin{equation} \nonumber
	\mu_2(x) = \frac{1}{1 + \left( \displaystyle \frac{x-20}{10} \right)^{10}}.
\end{equation}
Граничное условие $\mu_1(t) = \mu_2(-ct)$.

Реализовать составную схему, чисто неявную схему \eqref{c7eq4} и схему с полусуммой \eqref{c7eq5}. Положить на один график точное решение задачи $\mu_2(x-ct)$, а также результаты расчета по всем трем схемам. Объяснить поведение кривых численного решения.
