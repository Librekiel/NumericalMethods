\section{Задание к семинару №11}
Решить эллиптическое уравнение 
\begin{equation} \label{c11eq1}
	\begin{cases}
		\displaystyle \frac{\partial^2 u}{\partial x^2} + \frac{\partial^2 u}{\partial y^2} = 0 \\
		u(x,y) = \left. \mu(x,y) \right|_{(x,y) \in \Gamma}
	\end{cases}
\end{equation}
в области $(x,y) \in [0; a] \times [0; b]$ методом счета на установления с логарифмическим набором шагов. Здесь множество $\Gamma$ - граница расчетной области. Функцию $\mu(x,y)$, размеры области и параметры сетки взять из задания к предыдущему семинару.

Необходимо построить график невязки (левой части \eqref{c11eq1}) в норме $C$ (максимум модуля) 
от номера итерации. В конце построить результирующий трехмерный профиль.

Параметр $\varepsilon$, необходимый для построения логарифмического набора шагов, взять равным $10^{-12}$.

\subsection{Указания}
Для решения этой задачи удобно модифицировать программу с прошлого семинара. Фактически задача сводится к построению логарифмического набора шагов. Для этого необходимо определить границы спектра, вычислить необходимое число шагов и собственно построить логарифмический набор по известным границам спектра и числу шагов в наборе. Следует иметь в виду, что в двумерном случае следует взять половину того числа шагов, которое дает оценочная формула для одномерного случая. Для обеспечения соблюдения граничных условий в качестве начального приближения для $u(x,y)$ удобно взять $\mu(x,y)$.


